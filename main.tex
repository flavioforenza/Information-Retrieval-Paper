\documentclass[letterpaper,12pt]{article}
\usepackage{cite}
\usepackage{graphicx}
\usepackage{amsmath}
\usepackage{adjustbox}
\usepackage{multirow}
\usepackage{caption}
\usepackage{subcaption}
\newcommand{\RN}[1]{%
  \textup{\uppercase\expandafter{\romannumeral#1}}%
}
\captionsetup[figure]{labelsep=period}
\captionsetup[subfigure]{labelformat=simple} % default is 'parens'
\renewcommand\thesubfigure{\thefigure.\alph{subfigure}.}
\DeclareMathOperator*{\argmax}{argmax}
\DeclareMathOperator*{\argmin}{argmin}
\newtheorem{corollary}{CorollaURry}

\begin{document}
    
\title{\bfseries{Query expansion using Language Models}}
\author{Flavio Forenza}
\date\today
\maketitle

\begin{figure}[h!]
  \centering
  \includegraphics[width=0.2\linewidth]{images/logoUnimi2.png}
  \centering
\end{figure}

\begin{abstract}
  The use of different methods of language modeling, within the field 
  of information retrieval, is finding a wide diffusion in the state of the 
  art. Based on the accuracy of the language model, the problem related 
  to the information retrieval, in a large corpus of documents, can be 
  solved. In order to do this, the basic idea of these approaches is to 
  estimate a probabilistic linguistic model, for each document in the 
  collection, which is able to generate a ranking of relevant documents 
  given a query. One of the problems that afflicts this family of methods 
  is due to the lack of data present. From this, it is necessary to apply 
  smoothing techniques capable of adjusting the maximum likelihood 
  estimator in order to correct the generated imprecision. This paper 
  shows how their application outperforms the performance of classic 
  methods, such as \emph{tf-idf}, useful for generating rankings of 
  documents ordered by relevance. Using them, we will look at some concepts that 
  are useful for query expansion.
\end{abstract}

\newpage
\section*{INTRODUCTION}
Over the years, query expansion techniques have been proposed as a solution 
to the problem of term mismatches between a query and its relevant documents. 
There are typically two types of query expansion method families: 
Local (based on Pseudo / Relevance / Indirect Feedback) and Global (based 
on the generation and use of a thesaurus) \cite{01}. This paper focuses on the 
first category. Given the difficulty in gathering the users' feedback, only the 
first documents recovered will be considered relevant. Pseudo-relevant documents 
are used to find possible candidate terms to help expand the query \cite{02}.
This method has been further developed within the concept of the Language
Model\cite{06}. Statistical language models are widely used within Information 
Retrieval as they have a solid theoretical background and good empirical performance. 
At the state of the art, two well known probabilistic approaches 
in infromation retrieval are the Robertson and Sparck Jones model \cite{07} and 
the Croft and Harper model \cite{08}. Both models estimate the probability of relevance 
of each document to the query. Clearly, the two main problems relate 
to correctly estimating both the query model and the document model. A 
Language Model calculates the relevance of a document \emph{d} to a query \emph{q} by 
estimating a factored form of the distribution $P (q, D)$ \cite{03}. The construction 
of a good Language model must necessarily make use of smoothing models 
when one or more terms do not appear in a document. In the latter case, the 
maximum likelihood estimator would produce a probability equal to zero, 
invalidating the creation of the model itself \cite{04}\cite{05}. Another concept, useful 
for expanding the query and widely used within the project, is that of 
\emph{Word Embeddings}. The latter is obtained precisely from the use of Language 
models, or rather thanks to the co-occurrence of the terms made available. 
This method is based on being able to map every single word into a vector of 
real numbers, within a vector space. The idea is to be able to compare the 
distance of these in order to understand their similarity relationship. If one 
word is similar to another, then these will be considered as synonyms. The 
remainder of the paper is organized as follows. In the next section, \emph{Research 
question and Methodology}, the objectives of the project will be introduced 
followed by an overview of the proposed approach. The third section, \emph{Experimental 
result}, describes the whole system and the results obtained. Finally, 
the conclusions are presented in section four, \emph{Concluding remarks}.

\newpage
\section*{Research question and Methodology}

Language models can be used for a variety of purposes, such as: speech recognition, 
spelling correction, grammar correction and automatic translation. 
All these applications have the task of assigning a probability to a 
sequence of words, based on the number of times they appear in one or more 
documents. As briefly mentioned in the introductory chapter, the purpose 
of the following project is to verify the existence of a further method, which 
makes use of the concept of language model, specifically an \emph{n-grams} model, 
capable of expanding a query. Achieving this goal means solving the problem 
of mismatch between the terms present in the query and those present in a 
corpus of documents. The probability of generating a new \emph{q} query given the 
estimate of a Language Model for a \emph{D} document can only occur through a 
ranking of relevant documents. If the corpus of documents is large, thinking 
of generating \emph{n} Language models, with \emph{n} equal to the number of documents, 
turns out to be a computationally expensive operation. This paper has used 
a useful approach to be able to generate a first ranking of documents ordered 
by relevance with the query \emph{q}. The technique applied is the \emph{tf-idf} recovery 
model. The adoption of this method of weighing the terms, as well as being 
widely used in the state of the art, produces excellent results. The introduction 
of the \emph{LM}, and of other semantic analysis techniques, made it possible 
to outperform the performance of the \emph{tf-idf} baseline, generating ranking, 
starting from the latter, of documents more pertinent to the query \emph{q} \cite{09}. In 
the following paper, tests are carried out which certify the veracity of this 
thesis. It should be noted that the generation of the ranking, obtained from 
the weights of the \emph{tf-idf} method, is obtained using the well-known \emph{cosine 
similarity} metric between the weight vectors. To comply with the set objective, 
the position of the relevant target document, that is the document 
that the user is searching for, was kept track. This was possible because a 
query was chosen, among those available, that was close to the title of 
this document. The score assigned to the target document will represent the 
minimum \emph{threshold} to be able to form the new ranking of documents. This 
step is fundamental as, by setting a higher threshold, the target document 
would be lost in subsequent calculations. By setting a lower threshold, however, those documents that represent noise will be taken into account.

\newpage
\bibliographystyle{abbrv}
\bibliography{Bibliography}

\end{document}