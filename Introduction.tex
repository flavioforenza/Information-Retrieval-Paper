\section*{INTRODUCTION}
Over the years, query expansion techniques have been proposed as a solution 
to the problem of term mismatches between a query and its relevant documents. 
There are typically two types of query expansion method families: 
Local (based on Pseudo / Relevance / Indirect Feedback) and Global (based 
on the generation and use of a thesaurus) \cite{01}. This paper focuses on the 
first category. Given the difficulty in gathering the users' feedback, only the 
first documents recovered will be considered relevant. Pseudo-relevant documents 
are used to find possible candidate terms to help expand the query \cite{02}.
This method has been further developed within the concept of the Language
Model\cite{06}. Statistical language models are widely used within Information 
Retrieval as they have a solid theoretical background and good empirical performance. 
At the state of the art, two well known probabilistic approaches 
in infromation retrieval are the Robertson and Sparck Jones model \cite{07} and 
the Croft and Harper model \cite{08}. Both models estimate the probability of relevance 
of each document to the query. Clearly, the two main problems relate 
to correctly estimating both the query model and the document model. A 
Language Model calculates the relevance of a document \emph{d} to a query \emph{q} by 
estimating a factored form of the distribution $P (q, D)$ \cite{03}. The construction 
of a good Language model must necessarily make use of smoothing models 
when one or more terms do not appear in a document. In the latter case, the 
maximum likelihood estimator would produce a probability equal to zero, 
invalidating the creation of the model itself \cite{04}\cite{05}. Another concept, useful 
for expanding the query and widely used within the project, is that of 
\emph{Word Embeddings}. The latter is obtained precisely from the use of Language 
models, or rather thanks to the co-occurrence of the terms made available. 
This method is based on being able to map every single word into a vector of 
real numbers, within a vector space. The idea is to be able to compare the 
distance of these in order to understand their similarity relationship. If one 
word is similar to another, then these will be considered as synonyms. The 
remainder of the paper is organized as follows. In the next section, \emph{Research 
question and Methodology}, the objectives of the project will be introduced 
followed by an overview of the proposed approach. The third section, \emph{Experimental 
result}, describes the whole system and the results obtained. Finally, 
the conclusions are presented in section four, \emph{Concluding remarks}.