\section*{Concluding Remarks}
Coming to the conclusions, we can say that the generation of new queries happens correctly. However, there are some points that need to be improved, such as determining a method that sets a threshold useful for generating each ranking of relevant documents. During the project, the thresholds returned by the '$precision\_recall\_curve$' method of the sklearn library were taken into consideration. Unfortunately, not all of these had a useful score to be able to take into consideration the target document. It was therefore decided to take into consideration the initial threshold, referred to the weight, which the method of cosine similarity assigned to the target document. In addition, to speed up searches, an ad hoc threshold was set to prevent the ranking of relevant documents from containing all the documents in the collection. As for the use of the different parsers, as a future development it would be interesting to develop a hybrid system, capable of generating a high number of queries with low perplexity with the target document. In addition to this, it would be interesting to adopt a method capable of expanding the original query with the terms present in a vocabulary such as Wordnet instead of using the terms already present in the entire corpus of documents.