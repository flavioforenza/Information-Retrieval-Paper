\section*{Research question and Methodology}

Language models can be used for a variety of purposes, such as: speech recognition, 
spelling correction, grammar correction and automatic translation. 
All these applications have the task of assigning a probability to a 
sequence of words, based on the number of times they appear in one or more 
documents. As briefly mentioned in the introductory chapter, the purpose 
of the following project is to verify the existence of a further method, which 
makes use of the concept of language model, specifically an \emph{n-grams} model, 
capable of expanding a query. Achieving this goal means solving the problem 
of mismatch between the terms present in the query and those present in a 
corpus of documents. The probability of generating a new \emph{q} query given the 
estimate of a Language Model for a \emph{D} document can only occur through a 
ranking of relevant documents. If the corpus of documents is large, thinking 
of generating \emph{n} Language models, with \emph{n} equal to the number of documents, 
turns out to be a computationally expensive operation. This paper has used 
a useful approach to be able to generate a first ranking of documents ordered 
by relevance with the query \emph{q}. The technique applied is the \emph{tf-idf} recovery 
model. The adoption of this method of weighing the terms, as well as being 
widely used in the state of the art, produces excellent results. The introduction 
of the \emph{LM}, and of other semantic analysis techniques, made it possible 
to outperform the performance of the \emph{tf-idf} baseline, generating ranking, 
starting from the latter, of documents more pertinent to the query \emph{q} \cite{09}. In 
the following paper, tests are carried out which certify the veracity of this 
thesis. It should be noted that the generation of the ranking, obtained from 
the weights of the \emph{tf-idf} method, is obtained using the well-known \emph{cosine 
similarity} metric between the weight vectors. To comply with the set objective, 
the position of the relevant target document, that is the document 
that the user is searching for, was kept track. This was possible because a 
query was chosen, among those available, that was close to the title of 
this document. The score assigned to the target document will represent the 
minimum \emph{threshold} to be able to form the new ranking of documents. This step 
is fundamental as, by setting a higher threshold, the target document would 
be lost in subsequent calculations. By setting a lower threshold, however, 
those documents that represent noise will be taken into account. From this 
ranking, the \emph{LMs} for each document will be calculated \cite{10}, generating \emph{n} LMs, 
with \emph{n} equal to the number of relevant documents, based on the terms in the 
query. It should be noted that, in order to carry out this step, the concept 
of \emph{skip-gram} has been applied to the query, with step \emph{s} equal to two. The 
creation of each LM is possible only after using one of the existing smoothing 
techniques. To prevent a linguistic model from assigning zero probability to 
an invisible event, i.e. when a term present in the query is not present in 
an LM, we should eliminate some probability mass from some more frequent 
events and give it to events that we haven't never seen. There 
are a variety of methods for smoothing, some of these are: \emph{Laplace (add-
one) smoothing}, \emph{Linear Interpolation smoothing}, \emph{add-k smoothing}, \emph{back-off 
smoothing} and \emph{Kneser-Ney smoothing}. Among these, the following paper 
has experimented the use of the first two smoothing methods, each of which 
will produce different results useful for achieving the final goal. The core of 
the algorithm lies in being able to derive the best ranking of relevant documents, 
using the initial query, through an iterative process, as s changes. 
This variation will lead to the generation of several LMs, each with step \emph{s}, 
with $s=\{2,3,\ldots,10\}$. At each iteration, a new ranking of documents will be generated, thanks to the calculation of the Maximum Likelihood Estimation \emph{MLE}, between the query \emph{q} and each document \emph{d}(\ref{MLE}).
\begin{eqnarray}\label{MLE}
    P(q|d) & \approx & P(q|M_d) \nonumber \\
           & \approx & \prod_i^n{P(w_i|w_{i-1})} \nonumber \\
           & \approx & \frac{count(w_i,w_{i-1})}{\sum_{j=1}^n count(w_j,w_{i-1})} \nonumber \\
           & = & \frac{count(w_i, w_{i-1})}{count(w_{i-1})}
\end{eqnarray}
Among all these nine rankings, after being sorted, the one that will have the 
target document in the highest position, close to the first, will be chosen. 
Also thanks to this, it will be possible to determine the lambda parameters 
present in the interpolation smoothing technique. By focusing on the latter 
technique, it was necessary to implement the concept suggested by \cite{11} as 
regards the calculation of linear interpolation. Instead of adding one to 
the probability calculation, as is well known in Laplace's smoothing (\ref{Laplace}), linear 
interpolation, recursively, calculates the MLE of order two (\emph{bi-grams}) (\ref{LinInt}), up to 
the order zero (\emph{zero-grams})(\ref{zero-grams}).
\begin{equation}\label{Laplace}
    P(w_i|w_{i-1}) = \frac{count(w_{i-1})+1}{count(w_{i-1})+|V|}
\end{equation}
\begin{equation}\label{LinInt}
    P(w_i|w_{i-1}) = \lambda P(q|M_d) + (1-\lambda)P(q|M_c)
\end{equation}
\begin{equation}\label{zero-grams}
    P(w_i) = \lambda \frac{1}{|V|} + (1-\lambda)P(w_i)
\end{equation}

where:
\begin{itemize}
    \item $\sum_i \lambda_i = 1$
    \item $M_d$: represents the language model of the single document;
    \item $M_c$: represents the language model of the entire collection of documents;
    \item $|V|$: represents the number of unique words within the corpus of documents.
\end{itemize}
It is always good to specify that, like the iterative process of steps \emph{s}, both $\lambda$ 
parameters also follow the same reasoning. The idea is to assign a range of 
numbers $\lambda = \{0.1, 0.2,\ldots,1\}$ to both values. In both smoothing techniques, 
the \emph{perplexity} level existing between the query word set $W = \{w_1, w_2,\ldots,w_N\}$ and the language model of 
the target document present in the ranking of relevant documents returned 
by the tf-idf model will be calculated(\ref{perplexity}) \cite{12}.
\begin{equation}\label{perplexity}
    pp(W) = \sqrt[n]{\prod_{i=1}^N\frac{1}{P(w_i|w_{i-1})}} 
\end{equation}
After obtaining the best ranking, the next step is based on building a \emph{term-
term matrix} \cite{12}, where the terms in question are both those of the query and 
those belonging to the language model of the entire collection of relevant 
documents present in the ranking. Within this matrix, the numbers of co-occurrences 
between all terms will be reported. On this type of matrix it is 
possible to apply the calculation of \emph{Positive Pointwise Mutual Information 
(PPMI)}. PPMI draws on the intuition that the best way to weigh the association 
between two words is to ask how much more the two words co-occur in 
our corpus than we would have a priori expected them to appear by chance. 
This measure derives from the calculation of the standard PMI which represents 
a measure of frequency between two events \emph{x} and \emph{y}, compared to what 
we would expect if they were independent (\ref{pmi}).
\begin{equation}\label{pmi}
    \emph{pmi(x,y)} = \log_2\frac{P(x,y)}{P(x)P(y)}
\end{equation}
The ratio gives us an estimate of how much more the two words co-occur 
than we expect by chance. PMI values can be positive, negative or infinite. 
Negative values, which imply that events occur less often than we would 
expect by chance, tend to be unreliable when we have documents consisting 
of few terms, as in our case. To solve this problem, the calculation of the 
PPMI is used which replaces negative values with zero(\ref{ppmi}).
\begin{equation}\label{ppmi}
    PPMI(x,y) = \max(\log_2\frac{P(x,y)}{P(x)P(y)}, 0)
\end{equation}
But the question is: why is the PPMI calculation used? The adoption of 
this is useful for being able to calculate the similarity between words, i.e. 
their synonymy, search for paraphrases, keep track of the change in meaning 
of words and to automatically discover the meanings of words in different 
corpora. To find the words most similar to those in the query, the cosine 
similarity is calculated on the first ten word vectors that have the highest 
positive PPMI values. Eventually, each token in the query will have a maximum 
of ten expansion terms. Thanks to this, we are already seeing how 
query expansion can be done. Moving on, there is a problem to solve before we can 
calculate the similarity of the cosine: the \emph{sparsity} of the matrix. In order to 
obtain a good similarity, relatively low at the computational level, another 
concept has been implemented: \emph{Singular Value Decomposition (SVD)}. The 
idea of applying SVD on a term-term matrix was proposed by \cite{13}. Switching 
from sparse to dense vectors allows for better similarity comparisons. The 
SVD allows to decompose the term-term matrix (A), of dimensions $txd$, into 
three matrices:
\begin{equation}
    A = USV^t
\end{equation}
Where:
\begin{enumerate}
    \item \emph{U}: matrix of dimension $txm$ where the columns represent the left 
    singular vectors of matrix A;
    \item \emph{S}: diagonal matrix of dimension $mxm$, containing the singular values 
    of matrix A;
    \item \emph{$V^t$}: transposed matrix, of dimensions $mxd$, where the columns represent 
    the right singular vectors of matrix A.
\end{enumerate}