\section{\centering{"Query expansion using Language Models"}}

\begin{frame}{INTRODUCTION}
    Nowadays, searching the web for information appears to be one of the 
    simplest operations to perform. The difficulty perceived by the user in 
    formulating a query has been gradually reduced by techniques capable of 
    guiding his writing towards a correct generation of a query. These 
    techniques allow to improve the performance of information search 
    systems.
\end{frame}

\begin{frame}{GOAL}
    The purpose of this project is to be able to experiment with the use of 
    one of the most famous techniques, already present at the state of the 
    art, able to "assist" the user in formulating a correct query: {\bfseries Language Modeling}. Query expansion can be done using this concept to return a corpus of relevant documents.
\end{frame}

\begin{frame}{DATASET DESCRIPTION}
    
\end{frame}

\begin{frame}{RANKING GENERATION}
    
\end{frame}

\begin{frame}{RANKING EVALUATION}
    
\end{frame}

\begin{frame}{LANGUAGE MODELS}
    
\end{frame}

\begin{frame}{SMOOTHING METHODS}
    
\end{frame}

\begin{frame}{CORE}
    
\end{frame}

\begin{frame}{TERM-TERM MATRIX}
    
\end{frame}

\begin{frame}{POSITIVE POINTWISE MUTUAL INFORMATIONS (PPMI)}
    
\end{frame}

\begin{frame}{SINGULAR VALUE DECOMPOSITION (SVD)}
    
\end{frame}

\begin{frame}{QUERY EXPANSION}
    
\end{frame}

\begin{frame}{PERPLEXITY}
    
\end{frame}

\begin{frame}{SYSTEM EVALUATION WITH DIFFERENT PARSERS}
    
\end{frame}

\begin{frame}{CONCLUSIONS}
    
\end{frame}







